% !TEX TS-program = pdflatex
% !TEX encoding = UTF-8 Unicode

\documentclass[12pt,addpoints]{exam} % use larger type; default would be 10pt
\usepackage[utf8]{inputenc} % set input encoding (not needed with XeLaTeX)
\usepackage{listings}
\usepackage{amssymb} % for \nmid


%%% Examples of Article customizations
% These packages are optional, depending whether you want the features they provide.
% See the LaTeX Companion or other references for full information.

%%% PAGE DIMENSIONS
\usepackage{geometry} % to change the page dimensions
\geometry{a4paper} % or letterpaper (US) or a5paper or....
% \geometry{margins=2in} % for example, change the margins to 2 inches all round
% \geometry{landscape} % set up the page for landscape
%   read geometry.pdf for detailed page layout information


% \usepackage[parfill]{parskip} % Activate to begin paragraphs with an empty line rather than an indent

%%% PACKAGES
\usepackage{booktabs} % for much better looking tables
\usepackage{array} % for better arrays (eg matrices) in maths
\usepackage{paralist} % very flexible & customisable lists (eg. enumerate/itemize, etc.)
\usepackage{verbatim} % adds environment for commenting out blocks of text & for better verbatim
\usepackage{subfig} % make it possible to include more than one captioned figure/table in a single float
\usepackage{amsthm} % for proof environment
% These packages are all incorporated in the memoir class to one degree or another...
\usepackage{amsfonts} % allows use of mathbb 
\usepackage{amsmath}
\usepackage{multicol}
\usepackage{graphicx} % support the \includegraphics command and options


%%% HEADERS & FOOTERS
%\usepackage{fancyhdr} % This should be set AFTER setting up the page geometry
%\pagestyle{fancy} % options: empty , plain , fancy
%\renewcommand{\headrulewidth}{0pt} % customise the layout...
%\lhead{}\chead{}\rhead{}
%\lfoot{}\cfoot{\thepage}\rfoot{}

%%% SECTION TITLE APPEARANCE
\usepackage{sectsty}
\allsectionsfont{\sffamily\mdseries\upshape} % (See the fntguide.pdf for font help)
% (This matches ConTeXt defaults)

%%% ToC (table of contents) APPEARANCE
\usepackage[nottoc,notlof,notlot]{tocbibind} % Put the bibliography in the ToC
\usepackage[titles,subfigure]{tocloft} % Alter the style of the Table of Contents
\renewcommand{\cftsecfont}{\rmfamily\mdseries\upshape}
\renewcommand{\cftsecpagefont}{\rmfamily\mdseries\upshape} % No bold!
\newcommand{\Dgrad}[1]{\textsf{grad}\,#1\;}
\newcommand{\del}[1] {\mathrm{d}#1\ }
\newcommand{\Q}{\mathbb{Q}}
\newcommand{\F}{\mathbb{F}}
\newcommand{\C}{\mathbb{C}}
\newcommand{\Z}{\mathbb{Z}}
\newcommand{\R}{\mathbb{R}}
\newcommand{\N}{\mathbb{N}}
\newcommand{\PP}{\mathbb{P}}
\newcommand{\arcsec}{\mbox{arcsec}}
\newcommand{\arccsc}{\mbox{arccsc}}
\newcommand{\arccot}{\mbox{arccot}}
\newcommand{\csch}{\mbox{csch}}
\newcommand{\sech}{\mbox{sech}}

%%% NEW ENVIRONMENTS
\newtheorem{theorem}{Theorem}
\newtheorem{definition}[theorem]{Definition}
\newtheorem{example}[theorem]{Example}
\newtheorem{proposition}[theorem]{Proposition}
\newtheorem{corollary}[theorem]{Corollary}
\newtheorem{lemma}[theorem]{Lemma}

%%% END Article customizations

%%% The "real" document content comes below...


\usepackage{color}
\lstset{numbers=left}
\begin{document}

\printanswers

\title{Physics C: Part I}
\maketitle



\begin{center}
    This exam has \numquestions\ questions for a total of \numpoints\
    points. You have {\bf 2 Hours} to complete this exam.
\end{center}

\begin{questions}

  \section{Motion in a plane}

  \question (\totalpoints\ points) A soccer player kicks a ball at
  an angle of $37^\circ$ above the horizontal with an initial speed
  of 20.0 m/s.
  \begin{parts}
    \part[4] Find the time $t$ at which the ball reaches the highest
    point of its trajectory.
    \begin{solution}
      $$v_y(t) = 20.0 \sin(37 t) - 9.8t$$
      $$t = 1.2\mbox{s}$$
    \end{solution}
    
    \part[2] What is the range of the ball assuming the ground is
    level.
    \begin{solution}
      $$R = \frac{v_0^2}{g} \sin  2\theta = 39\mbox{m}$$
    \end{solution}
    
    \part[4] Ignoring air resistance, what is the ball's vertical
    position when it is 10.0 m down range from its starting
    position?
    
    \begin{solution}
      $$x_x(t) = 10.0\mbox{m} = 20.0 \cos 37t$$
      $$t = .63\mbox{s}$$
      $$x_y(.63) = 5.6$$
    \end{solution}
    
  \end{parts}
  
  \question[1] Can an object have a zero velocity and still be
  accelerating?
  
  \begin{solution}
    Yes, if you throw a ball up at the vertex.
  \end{solution}
  
  \question[1] Can an object be increasing in speed as its
  acceleration decreases? 
  \begin{solution}
    Yes.
  \end{solution}
  
  \section{Force}
  
  \question[2] A 5 kg concrete block is lowered with a downward
  acceleration of 2.8 m/s$^2$ by means of a rope. The force of the
  block on the rope is:
  \begin{oneparchoices}
    \choice 14 N up
    \choice 14 N down
    \choice 35 N up
    \CorrectChoice 35 N down
    \choice 49 N up
  \end{oneparchoices}
  
  \question[2] The ``reaction'' force does not cancel the ``action''
  force beacuse:
  \begin{oneparchoices}
    \choice action force is greater
    \CorrectChoice they act on different bodies
    \choice they are in the same direction
    \choice the reaction force exists only after the action force is
    removed
    \choice reaction force is greater than the action force
  \end{oneparchoices}
  
  \question[2] A block slides down a frictionless plane which makes
  an angle of $30^\circ$ with the horizontal. The acceleration of
  the block (in cm/s$^2$) is:
  \begin{oneparchoices}
    \choice 980
    \choice 566
    \choice 849
    \choice zero
    \CorrectChoice 490
  \end{oneparchoices}
  
  \question[2] A 1000 kg elevator is rising and its speed is
  increasing at 3 m/s$^2$. The tension in the elevator cable is:
  \begin{oneparchoices}
    \choice 6800 N
    \choice 1000 N
    \choice 3000 N
    \choice 9800 N
    \CorrectChoice 12800 N
  \end{oneparchoices}
  
  \section{Force and Friction}
  
  \question[2] A box rests on a rough board 10 meters long. When one
  end of the board is slowly raised to a height of 6 meters above the
  other end, the box begins to slide. The coefficient of static
  friction is:
  \begin{oneparchoices}
    \choice 0.8
    \choice 0.25
    \choice 0.4
    \choice 0.6
    \CorrectChoice 0.75
  \end{oneparchoices}

  \question[2] A horizontal force of 12 N pushes a 0.50 kg block
  against a vertical wall. The block is initially at rest. If $\mu_s =
  0.6$ and $\mu_k = 0.80$, the acceleration of the block in m/s$^2$
  is:
  \begin{oneparchoices}
    \CorrectChoice 0
    \choice 9.4
    \choice 9.8
    \choice 14.4
    \choice 19.2
  \end{oneparchoices}

  \question[2] Why do raindrops fall with constant speed during the
  later stages of their descent?
  \begin{oneparchoices}
    \choice the gravitational force is the same for all drops
    \CorrectChoice air resistance just balances the force of gravity
    \choice the drops all fall from the same height
    \choice the force of gravity is negligible for objects as small as
    raindrops
    \choice gravity cannot increase the speed of a falling object to
    more than 32 ft/s.
  \end{oneparchoices}

  \question (\totalpoints\ points) A man drags a 80 kg crate across a
  floor by pulling a rope that makes an angle of 15$^\circ$ above the
  horizontal.
  \begin{parts}
    \part[5] If the coefficient of static friction is 0.50 what force
    must the man exert to start the crate moving.
    \begin{solution}
      $$F = f/\cos \theta = \frac{0.50 N}{\cos \theta} = \frac{0.50
        (mg - F_m \sin \theta)}{\cos \theta}$$
      $$F + \frac{0.5F \sin \theta}{\cos \theta} = \frac{0.5 mg}{\cos
        \theta}$$
      $$F = \frac{0.5 mg}{\cos \theta (1 + 0.5 \tan \theta)}$$
      $$= 357.89 N$$
    \end{solution}
    \part[5] If the coefficient of kinetic friction is 0.35 what is
    the acceleration of the crate just after it starts moving while
    the man is still exerting the force of part (a)?
    \begin{solution}
      $$a = 1.3 m/s^2$$
    \end{solution}
  \end{parts}
  
  \section{Work and Energy}

  \question (\totalpoints\ points) A 10.0 kg block slides 3.0 m down a
  plane that is inclined at an angle of $20^\circ$ to the
  horizontal. The coefficient of kinetic friction between the block
  and the plane is 0.10.
  \begin{parts}
    \part[1] Calculate the work done by the gravitational force acting
    on the block.

    \begin{solution}
      $$W_g = 10.0 Kg \cdot g \cdot 3.0 m \cdot \cos (70^\circ) = 101
      J = 100 J$$
    \end{solution}

    \part[1] Calculate the work done by the force of kinetic friction
    acting on the block.

    \begin{solution}
      $$f_K = .10 \cdot [10 Kg \cdot g \cdot \cos(20^\circ)] = 9.22 N
      = 9.2 N$$
      $$ W_f = 9.22 N \cdot 3.0 m \cos (180^\circ) = -27.7J = -28J$$
    \end{solution}

    \part[1] Calculate the work done by the normal force acting on the
    block.

    \begin{solution}
      $$N = 10.0 Kg \cdot g \cdot \cos (20^\circ) = 92.2 N$$
      $$W_N = 92 N \cdot 3.0 m \cos (90^\circ) = 0J$$
    \end{solution}

    \part[1] Calculate the total (net) work done on the block.

    \begin{solution}
      $$W_{N_{net}} = (F_g \cdot \sin (20^\circ) - f_k) \cdot 3.0 m =
      73 J$$
    \end{solution}

    \part[3] What is the speed of the block as it passes the end of
    the 3.0 m section assuming it was moving at 1.0 m/s as it passed
    the beginning of the 3.0 m section? {\bf Use the work-energy
      theorem.}
    
    \begin{solution}
      $$\Delta K + \Delta U + \Delta E = 28 J$$
      $$\Delta K = \frac{1}{2} \cdot 10.0 Kg \cdot V^2 - \frac{1}{2}
      \cdot 10.0 Kg \cdot (1.0 m/s)^2 = 5.0 KgV^2 - 5.0 J$$
      $$\Delta U = mgy = 10.0 Kg \cdot g \cdot (\sin 20^\circ \cdot
      3.0 m) = -100.6 J$$
      $$5.0 KgV^2 - 5.0 J - 100.6 J + 28 J = 0$$
      $$V = 3.9 m/s$$
    \end{solution}

  \end{parts}

  \question (\totalpoints\ points) A 2.5 Kg block collides with a
  horizontal massless spring whose force constant is 320 N/m. The
  block compresses the spring a distance of 7.5 cm while coming to
  rest. The coefficient of kinetic friction between the block and the
  horizontal surface is 0.25.

  \begin{parts}

    \part[2] How much work was done by the spring in bringing the
    block to rest?

    \begin{solution}
      $$W = -\frac{1}{2}kx^2 = -0.90 J$$
    \end{solution}

    \part[2] How much work was done by the force of friction while the
    block was brought to rest.

    \begin{solution}
      $$f_k = .25 \cdot 2.5 Kg \cdot g = 6.13N$$
      $$W_{f_K} = 6.13 N \cdot - 0.75 m = -0.46 J$$
    \end{solution}

    \part[2] What was the speed of the block just as it hit the
    spring.

    \begin{solution}
      $$\Delta U = (-.9 J - .46 J) - 0J = -1.36 J$$
      $$\Delta K = (1.36 - 0J) = \frac{1}{2} 2.5 Kg v^2$$
      $$v = 1.04 m/s = 1.0 m/s$$
    \end{solution}

  \end{parts}

\section{Particles and Collisions}

\question[2] Two skaters, one with mass 65 kg and the other with mass
40 kg stand on an ice rink holding a pole with a length of 10 m and a
mass that is negligible. Starting from the ends of the pole, the
skaters pull themselves along the pole until they meet. How far will
the 40-kg skater move?

\begin{solution}
  6.19 m
\end{solution}

\question[2] Find the momentum and the kinetic energy of a 4-g bullet
with a speed of 950 m/s.

\begin{solution}
  momentum is 3.80 kg $\cdot$ m/s and the energy is 1810 J.
\end{solution}

\question[2] A cannon whose mass $M$ is 1300 kg fires a 72-kg ball in
a horizontal direction with a muzzle speed $v$ of 55 m/s. The cannon
is mounted so that it can recoil freely. What is the velocity $V$ of
the recoiling cannon with respect to the earth?

\begin{solution}
  $$P = MV + m(v + V)$$
  $$V = -\frac{mv}{M + m} = -2.9 m/s$$
\end{solution}

\question[4] A 140 g baseball in horizontal flight with a speed $v_i$
of 39 m/s is struck by a batter. After leaving the bat, the ball
travels in the opposite direction with a speed $v_j$, also 39
m/s. What is the impulse $J$ acted on the ball while it was in contact
with the bat? If the impact time $\delta t$ for the baseball-bat
collision is 1.2 ms, what average force acts on the baseball?

\begin{solution}
  $$J = p_f - p_i = mv_f - mv_i = 10.9 kg m/s$$
  $$F = \frac{J}{\Delta t} = 9100 N$$
\end{solution}

\section{Rotation}

\question (\totalpoints\ points) A body rotates about a fixed axis
with constant angular acceleration of -20 rad$/s^2$. If at $t = 0$ its
angular speed is $+65$ rad/s then

\begin{parts}
  \part[2] What is its angular speed at $t = 2.5s$?
  \begin{solution}
    $$\omega = 15 rad/s$$
  \end{solution}
  \part[2] Through what angle has the body turned in 2.5 seconds?
  \begin{solution}
    $$0 = \omega_0 t + \frac{1}{2} \alpha t^2 = 100 rad$$
  \end{solution}
  \part[2] If the body is a disc with a radius of 2.0 m what is the
  tangential acceleration of a particle on the rim at $t = 2.5$
  seconds?
  \begin{solution}
    $a_t = \alpha r = -40 m/s^2$ tangent to the path
  \end{solution}
  \part[2] What is the radial acceleration of the same particle at the
  same time?
  \begin{solution}
    $a_r = -\omega^2 r = -450 m/s^2$ toward the center.
  \end{solution}
\end{parts}

\question(\totalpoints\ points) A constant torque of 30.0 mN is
exerted for 5.0 seconds on a disc initially at rest. The frequency of
rotation at the end of this time period is 120 rev/min.
\begin{parts}
  \part[2] What is the rotational inertia of the disc?
  \begin{solution}
    $$\alpha = 2.52 rad/s^2$$
    $$I = 11.9 Kgm^2$$
  \end{solution}
  \part[2] If the disc has a radius of 60.0 cm what is its mass?
  \begin{solution}
    $$I = \frac{MR^2}{2}$$
    $$M = 66 Kg$$
  \end{solution}
  \part[2] If the force associated with the torque is applied tangent
  to the rim of the disc, what is the size of this force?
  \begin{solution}
    $${\bf \tau} = rF\sin \theta$$
    $$F = 50 N$$
  \end{solution}
  \part[3] What power is delivered to the disc?
  \begin{solution}
    $$\Delta K = \frac{1}{2}I\omega^2 = 945 J$$
    $$P = 189 W$$
  \end{solution}
\end{parts}

\question[3] The rotational inertia of a hollow spherical shell of
mass 10.0 kg and radius 2.0 m about an axis through its center is 26.7
Kg m$^2$. What is its rotational inertia about an axis tangent to its
outer shell?
  \begin{solution}
    Use the parallel-Axis theorem:
    $$I = I_{cm} + Mh^2 = 66.7 Kg m^2$$
  \end{solution}



\section{Torque and angular momentum}

\question (\totalpoints\ points) At a certain time the position vector
in meters of a 0.25 kg object is ${\bf r} = 2.0 {\bf i} - 2.0 {\bf
  k}$. At that instant its velocity in meters per second is ${\bf v} =
-5.0{\bf i} + 5.0{\bf k}$ and the force in newtons acting on it is
${\bf F} = 4.0{\bf j}$.

\begin{parts}
  \part[2] What is its angular momentum relative to the origin?
  \begin{solution}
    $${\bf l} = {\bf r} \times {\bf p} = 0 Nms$$
  \end{solution}
  \part[2] What torque act on the object?
  \begin{solution}
    $${\bf \tau} = {\bf r} \times {\bf F} = 8{\bf i} + 8{\bf k} Nm$$
  \end{solution}
  \part[2] Asuming the torque is a constant for the next 1.0 second
  what is the angular momentum of the object relative to the origin at
  the end of 1.0 seconds?
  \begin{solution}
    $${\bf l_1} = 8{\bf i} + 8{\bf k} Nms$$
  \end{solution}
\end{parts}

\question (\totalpoints\ points) A student sits on a stool that is
free to rotate about a vertical axis. He holds his arms extended
horizontally with a 4.0 kg weight in each hand. He is rotating with an
angular speed of 0.50 rev/s. The student's rotational inertia is 5.0kg
m$^2$. When he lowers his arms to his side his rotational inertia is
now 4.9 kgm$^2$. {\it This rotational inertia does not include the
  weights.} The original distance of each weight from the axis of
rotation is 90.0 cm and their distance after he lowers his arms is
15cm from the axis.

\begin{parts}
  \part[4] What is the angular speed of the student after he lowers
  his arms?
  \begin{solution}
    $$l_{out} = I\omega = (5.0 Kgm^2 + 2\cdot 4.0Kg \cdot 0.900^2
    m)(3.14 rad/s) = 36.0 Nms$$
    $$l_{in} = (4.9Kgm^2 + 2 \cdot 4.0 Kg \cdot 0.15^2 m)\omega_{in}$$
    $$omega_in = 7.09 rad/s$$
  \end{solution}

  \part[4] Is kinetic energy conserved as the student pulls in his
  arms? Prove your answer.
  \begin{solution}
   $$\Delta K = \frac{1}{2} 5.08Kg m^2 \cdot 7.09 rad/s - \frac{1}{2}
    11.48 Kgm^2 \cdot 3.14^2 rad/s \ne 0$$
    No, $K$ is not conserved.
  \end{solution}

\end{parts}

\section{Oscillations}

\question[2] An object attached to one end of a spring makes 20
vibrations in 10 seconds. Its period is:
\begin{oneparchoices}
  \choice 20 Hz
  \choice 10 s
  \choice 0.05 Hz
  \choice 2 s
  \CorrectChoice 0.50 s
\end{oneparchoices}

\question[2] In simple harmonic motion, the restoring force must be
proportional to the:
\begin{oneparchoices}
  \choice amplitude
  \choice frequency
  \choice velocity
  \CorrectChoice displacement
  \choice displacement squared
\end{oneparchoices}

\question[2] A simple pendulum is suspended from the ceiling of an
elevator. The elevator is accelerating upwards with acceleration
$a$. The period of this pendulum in terms of its length $L$, $g$ and
$a$ is:
\begin{oneparchoices}
  \choice $2\pi\sqrt{L/g}$
  \CorrectChoice $2\pi\sqrt{L/(g+a)}$
  \choice $2\pi\sqrt{L/(g - a)}$
  \choice $2\pi \sqrt{L/a}$
  \choice $\frac{1}{2}\pi \sqrt{g/L}$
\end{oneparchoices}

\question[2] A harmonic oscillatory (mass $m$, spring constant $k$)
has amplitue $A$. It's maximum speed is?
\begin{solution}
  $$\sqrt{\frac{k}{m}} A$$
\end{solution}

\question[2] An 0.200 kg mass attached to a spring whose force
constant is 500 N/m executes simple harmonic motion with amplitude
0.100 m. Its maximum speed is:
\begin{oneparchoices}
  \choice 25 m/s
  \CorrectChoice 5 m/s
  \choice 1 m/s
  \choice 15.8 m/s
  \choice 0.2 m/s
\end{oneparchoices}

\question[2] A block, attached to a apring undergoes simple harmonic
motion on a horizontal frictionless surface. Its total energy is 50
J. When the displacement is half the amplitude, the kinetic energy is:
\begin{oneparchoices}
  \choice 0
  \choice 12.5 J
  \choice 25 J
  \CorrectChoice 37.5 J
  \choice 50 J
\end{oneparchoices}

\question[2] A particle moves in simple harmonic motion according to
$x = 2\cos(50 t)$, where $x$ is in meters and $t$ is in seconds. Its
maximum velocity  (in m/s) is:
\begin{oneparchoices}
  \choice $100 \sin (50 t)$
  \choice $100 \cos (50 t)$
  \CorrectChoice 100
  \choice 200
  \choice none of these
\end{oneparchoices}

\question[2] A mass-spring system is oscillating with amplitude
$A$. The kinetic energy will equal the potential energy only when the
displacement is
\begin{oneparchoices}
  \choice zero
  \choice $A/4$
  \CorrectChoice $A/\sqrt{2}$
  \choice $A/2$
  \choice all between $-A$ and $+A$.
\end{oneparchoices}

\section{Waves}

\question[2] A sinusoidal wave travels along a string. The time for a
particular point to move from maximum displacement to zero is
0.17s. The wavelength is 1.4 m. What are the period, frequency and wave speed?
\begin{solution}
  Period = 0.68 s, Frequency = 1.47 Hz, and wave speed is 2.06 m/s.
\end{solution}

\question[2] A climber whose mass $m$ is 86 kg slides down a rope. The
leader wishs to signal him by giving the top end of the rope a sharp
tap. How long does it take for the signal to travel 32 m down the
rope? The linear density $\mu$ of the rope is 74 g/m.
\begin{solution}
  $$v = \sqrt{\frac{\tau}{\mu}} = 107 m/s.$$
\end{solution}

\question[2] A 125-cm length of string has a mass of 2.0g. It is
streched with tension of 7.0 N between fixed supports. What is the
wave speed for this string? What is the lowest resonant frequency?
\begin{solution}
  The wave speed is 66.1 m/s and the lowest resonant frequency is 26.4 Hz.
\end{solution}




\end{questions}

\end{document}
