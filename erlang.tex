% !TEX TS-program = pdflatex
% !TEX encoding = UTF-8 Unicode

\documentclass[12pt,addpoints]{exam} % use larger type; default would be 10pt
\usepackage[utf8]{inputenc} % set input encoding (not needed with XeLaTeX)
\usepackage{listings}
\usepackage{amssymb} % for \nmid


%%% Examples of Article customizations
% These packages are optional, depending whether you want the features they provide.
% See the LaTeX Companion or other references for full information.

%%% PAGE DIMENSIONS
\usepackage{geometry} % to change the page dimensions
\geometry{a4paper} % or letterpaper (US) or a5paper or....
% \geometry{margins=2in} % for example, change the margins to 2 inches all round
% \geometry{landscape} % set up the page for landscape
%   read geometry.pdf for detailed page layout information


% \usepackage[parfill]{parskip} % Activate to begin paragraphs with an empty line rather than an indent

%%% PACKAGES
\usepackage{booktabs} % for much better looking tables
\usepackage{array} % for better arrays (eg matrices) in maths
\usepackage{paralist} % very flexible & customisable lists (eg. enumerate/itemize, etc.)
\usepackage{verbatim} % adds environment for commenting out blocks of text & for better verbatim
\usepackage{subfig} % make it possible to include more than one captioned figure/table in a single float
\usepackage{amsthm} % for proof environment
% These packages are all incorporated in the memoir class to one degree or another...
\usepackage{amsfonts} % allows use of mathbb 
\usepackage{amsmath}
\usepackage{multicol}
\usepackage{graphicx} % support the \includegraphics command and options


%%% HEADERS & FOOTERS
%\usepackage{fancyhdr} % This should be set AFTER setting up the page geometry
%\pagestyle{fancy} % options: empty , plain , fancy
%\renewcommand{\headrulewidth}{0pt} % customise the layout...
%\lhead{}\chead{}\rhead{}
%\lfoot{}\cfoot{\thepage}\rfoot{}

%%% SECTION TITLE APPEARANCE
\usepackage{sectsty}
\allsectionsfont{\sffamily\mdseries\upshape} % (See the fntguide.pdf for font help)
% (This matches ConTeXt defaults)

%%% ToC (table of contents) APPEARANCE
\usepackage[nottoc,notlof,notlot]{tocbibind} % Put the bibliography in the ToC
\usepackage[titles,subfigure]{tocloft} % Alter the style of the Table of Contents
\renewcommand{\cftsecfont}{\rmfamily\mdseries\upshape}
\renewcommand{\cftsecpagefont}{\rmfamily\mdseries\upshape} % No bold!
\newcommand{\Dgrad}[1]{\textsf{grad}\,#1\;}
\newcommand{\del}[1] {\mathrm{d}#1\ }
\newcommand{\Q}{\mathbb{Q}}
\newcommand{\F}{\mathbb{F}}
\newcommand{\C}{\mathbb{C}}
\newcommand{\Z}{\mathbb{Z}}
\newcommand{\R}{\mathbb{R}}
\newcommand{\N}{\mathbb{N}}
\newcommand{\PP}{\mathbb{P}}
\newcommand{\arcsec}{\mbox{arcsec}}
\newcommand{\arccsc}{\mbox{arccsc}}
\newcommand{\arccot}{\mbox{arccot}}
\newcommand{\csch}{\mbox{csch}}
\newcommand{\sech}{\mbox{sech}}

%%% NEW ENVIRONMENTS
\newtheorem{theorem}{Theorem}
\newtheorem{definition}[theorem]{Definition}
\newtheorem{example}[theorem]{Example}
\newtheorem{proposition}[theorem]{Proposition}
\newtheorem{corollary}[theorem]{Corollary}
\newtheorem{lemma}[theorem]{Lemma}

%%% END Article customizations

%%% The "real" document content comes below...


\usepackage{color}
\lstset{numbers=left}
\begin{document}

\printanswers

\title{Erlang}
\maketitle



\begin{center}
    This exam has \numquestions\ questions for a total of \numpoints\
    points.You have {\bf 12 Hours} to complete this exam.
\end{center}

% 24 lectures, 12 hours, so 30 points per lecture.

\begin{questions}

    % Lecture 1 question
    \question[5] Discuss, very briefly, what failure mode you should
    use when coding in Erlang.

    \begin{solution}
        One Erlang process crashing doesn't automatically crash other
        processes.  Thus local crashes do not lead to global
        crashes. Further it is possible for processes to monitor other
        processes.  Erlang's OTP comes with servers, state machines in
        the library for use.
    
        In general the Erlang style is to *not* code defensively, but
        to code for the correct case, let the process fail!
    \end{solution}

    % Lecture 2 question. This is exercise 2-1 in C & T, plus some extra.
    \question (\totalpoints\ points) What is the result of entering
    the following into the Erlang shell?

    \begin{parts}
        \part[1] -16\#EA.
        \begin{solution}
            -234
        \end{solution}
        \part[1] 3 + 2\#10101
        \begin{solution}
            An error. Shell entries end with .          
        \end{solution}
        \part[2] 2\#1010 + \$A.
        \begin{solution}
            75
        \end{solution}
        \part[1] 1.41 + -1.234E-1.
        \begin{solution}
            1.2866
        \end{solution}
        \part[1] 1 + 3 / 2.
        \begin{solution}
            2.5
        \end{solution}
        \part[1] {\tt is\_boolean( is\_boolean( 1510 ) ) < false.}
        \begin{solution}
            {\tt false}
        \end{solution}
        \part[1] {\tt not((1<3) xor (2==2)).}
        \begin{solution}
            {\tt true}
        \end{solution}
    \end{parts}

    % Lecture 2 question. 
    \question[3] Write two expression to enter in the Erlang shell that
    will compute m and n where 1510 = 214m + n and $0 \le n < 214$.
    \begin{solution}
        For m use {\tt 1510 div 214.} For n use {\tt 1510 rem 214.}
    \end{solution}

    % Lecture 3 question.
    \question[4] Construct a tuple to store the name Joe
    Armstrong. Discuss the reasons for it's construction.
    \begin{solution}        
        {\tt \{person, 'Joe', 'Armstrong'\} }
        
        Points:
        \begin{enumerate}
        \item The person atom serves as a tag. Sort of like a hacked
        on type.
        \item We need to use the quotes around the Joe and Armstrong
        atoms because you cannot start an unquoted atom with an
        uppercase letter.
        \end{enumerate}
        
    \end{solution}

    % Lecture 3
    \question[4] Write an expression that sets the the first element
    the tuple {\tt \{a, boring, tuple\} } to the second to last
    element of the tuple {\tt\{abc, {def, 123}, 'Joe', hack\}}
    element, without using the knowledge that the second to last
    element of the tuple is 'Joe'.

    \begin{solution}
        {\tt setelement(1, \{a, boring, tuple\},
        element(tuple\_size(\{abc, \{def, 123\}, 'Joe', hack\}) - 1, \{abc,
        \{def, 123\}, 'Joe', hack\})) }
    \end{solution}

    % Lecture 3
    % some of this is from Cesarini & Thompson Exercise 2-1 part C.
    \question (\totalpoints\ points) What is the value of entering the
    following expressions into the Erlang shell?
    \begin{parts}
        \part[1] {\tt [\$H, 101, \$L, \$A + 11]. }
        \begin{solution}
            {\tt ``HeLL''}
        \end{solution}    
        \part[1] {\tt [\$H, 101, \$L, \$L, o]. }
        \begin{solution}
            {\tt [72, 101, 76, 76, o]}
        \end{solution}    
        \part[1] {\tt L = [1 | [2, 3]]. }
        \begin{solution}
            {\tt [1, 2, 3]}
        \end{solution}    
        \part[1] {\tt [[3, 2]|1]]. }
        \begin{solution}
            {\tt [[3, 2]|1]}
        \end{solution}    
        \part[2] Assume $L$ as above, {\tt [H|T] = L. }
        \begin{solution}
            This returns {\tt [1, 2, 3]} and sets {\tt H = [1]} and
            {\tt T = [2, 3]}.
        \end{solution}    
        \part[1] {\tt \{boo, hoo, new, sue\} < {boo, hoo, new, sue}. }
        \begin{solution}
            {\tt true}
        \end{solution}
        \part[1] {\tt 1 =/= 1.0.}
        \begin{solution}
            {\tt true}
        \end{solution}
    \end{parts}

    % Lecture 3
    \question[10] Explain what the append, subtract and string
    concatentation operators are, what they do, and give examples of
    using them.
    \begin{solution}
        1. Append: This is the {\tt ++} operators. Given two lists it
           joins them. For example:

           {\tt >[ [1,2], 3, 4] ++ [ [1, 2], 4, 5].

                 [[1,2],3,4,[1,2],4,5] }

       2. Subtract: This is the {\tt --} operator and removes every
          element of the second list from the first:

          {\tt >[ [1,2], 3, 4] -- [ [1, 2], 4, 5].

               [3] }

       3. String concatenation: writing strings next to each
          other. Remember strings are lists:

          {\tt >''Hello, `` ``World''.
              
               ``Hello, World'' }
    \end{solution}

   
\end{questions}

\end{document}
