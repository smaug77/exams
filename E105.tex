% !TEX TS-program = pdflatex
% !TEX encoding = UTF-8 Unicode

\documentclass[12pt,addpoints]{exam} % use larger type; default would be 10pt
\usepackage[utf8]{inputenc} % set input encoding (not needed with XeLaTeX)
\usepackage{listings}
\usepackage{amssymb} % for \nmid


%%% Examples of Article customizations
% These packages are optional, depending whether you want the features they provide.
% See the LaTeX Companion or other references for full information.

%%% PAGE DIMENSIONS
\usepackage{geometry} % to change the page dimensions
\geometry{a4paper} % or letterpaper (US) or a5paper or....
% \geometry{margins=2in} % for example, change the margins to 2 inches all round
% \geometry{landscape} % set up the page for landscape
%   read geometry.pdf for detailed page layout information


% \usepackage[parfill]{parskip} % Activate to begin paragraphs with an empty line rather than an indent

%%% PACKAGES
\usepackage{booktabs} % for much better looking tables
\usepackage{array} % for better arrays (eg matrices) in maths
\usepackage{paralist} % very flexible & customisable lists (eg. enumerate/itemize, etc.)
\usepackage{verbatim} % adds environment for commenting out blocks of text & for better verbatim
\usepackage{subfig} % make it possible to include more than one captioned figure/table in a single float
\usepackage{amsthm} % for proof environment
% These packages are all incorporated in the memoir class to one degree or another...
\usepackage{amsfonts} % allows use of mathbb 
\usepackage{amsmath}
\usepackage{multicol}
\usepackage{graphicx} % support the \includegraphics command and options


%%% HEADERS & FOOTERS
%\usepackage{fancyhdr} % This should be set AFTER setting up the page geometry
%\pagestyle{fancy} % options: empty , plain , fancy
%\renewcommand{\headrulewidth}{0pt} % customise the layout...
%\lhead{}\chead{}\rhead{}
%\lfoot{}\cfoot{\thepage}\rfoot{}

%%% SECTION TITLE APPEARANCE
\usepackage{sectsty}
\allsectionsfont{\sffamily\mdseries\upshape} % (See the fntguide.pdf for font help)
% (This matches ConTeXt defaults)

%%% ToC (table of contents) APPEARANCE
\usepackage[nottoc,notlof,notlot]{tocbibind} % Put the bibliography in the ToC
\usepackage[titles,subfigure]{tocloft} % Alter the style of the Table of Contents
\renewcommand{\cftsecfont}{\rmfamily\mdseries\upshape}
\renewcommand{\cftsecpagefont}{\rmfamily\mdseries\upshape} % No bold!
\newcommand{\Dgrad}[1]{\textsf{grad}\,#1\;}
\newcommand{\del}[1] {\mathrm{d}#1\ }
\newcommand{\Q}{\mathbb{Q}}
\newcommand{\F}{\mathbb{F}}
\newcommand{\C}{\mathbb{C}}
\newcommand{\Z}{\mathbb{Z}}
\newcommand{\R}{\mathbb{R}}
\newcommand{\N}{\mathbb{N}}
\newcommand{\PP}{\mathbb{P}}
\newcommand{\arcsec}{\mbox{arcsec}}
\newcommand{\arccsc}{\mbox{arccsc}}
\newcommand{\arccot}{\mbox{arccot}}
\newcommand{\csch}{\mbox{csch}}
\newcommand{\sech}{\mbox{sech}}

%%% NEW ENVIRONMENTS
\newtheorem{theorem}{Theorem}
\newtheorem{definition}[theorem]{Definition}
\newtheorem{example}[theorem]{Example}
\newtheorem{proposition}[theorem]{Proposition}
\newtheorem{corollary}[theorem]{Corollary}
\newtheorem{lemma}[theorem]{Lemma}

%%% END Article customizations

%%% The "real" document content comes below...


\usepackage{color}
\lstset{numbers=left}
\begin{document}

\printanswers

\title{Econ 105}
\maketitle



\begin{center}
    This exam has \numquestions\ questions for a total of \numpoints\
    points. You have {\bf 2 Hours} to complete this exam.
\end{center}

% Syllabus
%
% Microeconomics:
% -------------
%
% 1. scarcity, effiency, specialization
% 2. Supply and demand
% 3. Consumer demand
% 4. Supply: production, costs and profits
% 5. Perfect competition
% 6. Imperfect competition
% 7. Factor prices, incomes, and income distribution
% 8. market efficiency and market failures
%
% Macroeconomics
% -------------
%
% 9. Macro overview
% 10. Measuring total output and inflation
% 11. Short-run AE model of output
% 12. Fiscal policy and government budgets
% 13. Money, interest rate, and monetary policy
% 14. AD and AS
% 15. Unemployment and inflation
% 16. Economic growth, producivity and supply-side economics

\begin{questions}

  \section{Multiple Choice}

  % 2
  \question[2] At the equilibrium price for a good:
  \begin{choices}
    \choice prices will remain unchanged, even if there is excess
    demand.
    \choice there may be excess demand for the good but not excess
    supply.
    \choice shifts in the supply or demand curves will not cause price
    changes.
    \CorrectChoice the quantity supplied of the good must equal the
    quantity demanded.
  \end{choices}

  % 3
  \question[2] A consumer will be maximizing utility at
  \begin{choices}
    \choice any intersection of the budget line and an indifference
    curve
    \choice any point on the highest indifference curve shown on the
    indifference diagram.
    \CorrectChoice the point where the indifference curve is tangent
    to the budget line
    \choice any point on the budget line
    \choice any point inside the budget line.
  \end{choices}

  % 4
  \question[2] The price of coal rose and the quantity sold
  fell. Which of the following is consistent with this observation
  (everything else being equal)?
  \begin{choices}
    \choice the price of oil rose
    \choice the productivity of coal miners rose
    \CorrectChoice more costly coal mining equipment was installed
    \choice the population increased
  \end{choices}

  % 4
  \question[2] The law of increasing (opportunity) costs explains why
  the production possibilities frontier  (PPF) is
  \begin{choices}
    \choice downward sloping
    \choice upward sloping
    \choice bowed inward
    \CorrectChoice bowed outward
    \choice undefined because no market will exist in this case.
  \end{choices}

  % 4
  \question[2] Marginal cost is the
  \begin{choices}
    \choice total cost divided by the number of units produced.
    \choice variable cost divided by the number of units produced
    \choice extra average cost of producing an additional unit of
    output
    \CorrectChoice extra cost of producing an additional unit of
    output
  \end{choices}

  % 5
  \question[2] Which of the following is not characteristic of a
  perfectly competitive industry?
  \begin{choices}
    \CorrectChoice Price is the primary decision variable of
    individual firms in the industry
    \choice There are many buyers and sellers of the industry product
    \choice There are no barriers to exit from or entry to the
    industry
    \choice The products of all firms in the industry are identical
    \choice None of the above.
  \end{choices}

  \section{Short Answer}

  % 3 

\end{questions}

\end{document}
