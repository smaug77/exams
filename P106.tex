% !TEX TS-program = pdflatex
% !TEX encoding = UTF-8 Unicode

\documentclass[12pt,addpoints]{exam} % use larger type; default would be 10pt
\usepackage[utf8]{inputenc} % set input encoding (not needed with XeLaTeX)
\usepackage{listings}
\usepackage{amssymb} % for \nmid


%%% Examples of Article customizations
% These packages are optional, depending whether you want the features they provide.
% See the LaTeX Companion or other references for full information.

%%% PAGE DIMENSIONS
\usepackage{geometry} % to change the page dimensions
\geometry{a4paper} % or letterpaper (US) or a5paper or....
% \geometry{margins=2in} % for example, change the margins to 2 inches all round
% \geometry{landscape} % set up the page for landscape
%   read geometry.pdf for detailed page layout information


% \usepackage[parfill]{parskip} % Activate to begin paragraphs with an empty line rather than an indent

%%% PACKAGES
\usepackage{booktabs} % for much better looking tables
\usepackage{array} % for better arrays (eg matrices) in maths
\usepackage{paralist} % very flexible & customisable lists (eg. enumerate/itemize, etc.)
\usepackage{verbatim} % adds environment for commenting out blocks of text & for better verbatim
\usepackage{subfig} % make it possible to include more than one captioned figure/table in a single float
\usepackage{amsthm} % for proof environment
% These packages are all incorporated in the memoir class to one degree or another...
\usepackage{amsfonts} % allows use of mathbb 
\usepackage{amsmath}
\usepackage{multicol}
\usepackage{graphicx} % support the \includegraphics command and options


%%% HEADERS & FOOTERS
%\usepackage{fancyhdr} % This should be set AFTER setting up the page geometry
%\pagestyle{fancy} % options: empty , plain , fancy
%\renewcommand{\headrulewidth}{0pt} % customise the layout...
%\lhead{}\chead{}\rhead{}
%\lfoot{}\cfoot{\thepage}\rfoot{}

%%% SECTION TITLE APPEARANCE
\usepackage{sectsty}
\allsectionsfont{\sffamily\mdseries\upshape} % (See the fntguide.pdf for font help)
% (This matches ConTeXt defaults)

%%% ToC (table of contents) APPEARANCE
\usepackage[nottoc,notlof,notlot]{tocbibind} % Put the bibliography in the ToC
\usepackage[titles,subfigure]{tocloft} % Alter the style of the Table of Contents
\renewcommand{\cftsecfont}{\rmfamily\mdseries\upshape}
\renewcommand{\cftsecpagefont}{\rmfamily\mdseries\upshape} % No bold!
\newcommand{\Dgrad}[1]{\textsf{grad}\,#1\;}
\newcommand{\del}[1] {\mathrm{d}#1\ }
\newcommand{\Q}{\mathbb{Q}}
\newcommand{\F}{\mathbb{F}}
\newcommand{\C}{\mathbb{C}}
\newcommand{\Z}{\mathbb{Z}}
\newcommand{\R}{\mathbb{R}}
\newcommand{\N}{\mathbb{N}}
\newcommand{\PP}{\mathbb{P}}
\newcommand{\arcsec}{\mbox{arcsec}}
\newcommand{\arccsc}{\mbox{arccsc}}
\newcommand{\arccot}{\mbox{arccot}}
\newcommand{\csch}{\mbox{csch}}
\newcommand{\sech}{\mbox{sech}}

%%% NEW ENVIRONMENTS
\newtheorem{theorem}{Theorem}
\newtheorem{definition}[theorem]{Definition}
\newtheorem{example}[theorem]{Example}
\newtheorem{proposition}[theorem]{Proposition}
\newtheorem{corollary}[theorem]{Corollary}
\newtheorem{lemma}[theorem]{Lemma}

%%% END Article customizations

%%% The "real" document content comes below...


\usepackage{color}
\lstset{numbers=left}
\begin{document}

\printanswers

\title{Physics 106}
\maketitle



\begin{center}
    This exam has \numquestions\ questions for a total of \numpoints\
    points. You have {\bf 2 Hours} to complete this exam.
\end{center}

% 14 weeks:
% 1: 18 (8.5 points)
% 2: 19 (17)
% 3: 20 (25.5)
% 4: 21-22 (34)
% 5: 22-23 (42.5)
% 6: 24 (53)
% 7: 25 (61.5)
% 8: 26 (70)
% 9: 28 (78.5)
% 10: 29 (87)
% 11: 34 (95.5)
% 12: 30 (104)
% 13: 33 (112.5)
% 14: 31-32 (121)
% 8.5 points per week.
\begin{questions}

  \section{Discrete Charge Distributions}
  
  % 18.39
  \question[8] An electric dipole consists of a positive charge $q$ on
  the $x$ axis at $x = a$ and a negative charge $-q$ on the $x$ axis
  at $x = -a$. Find the magnitude and direction of the electric field
  at a point $y$ on the $y$ axis, and show that for $y >> a$, the
  field is approximately ${\bf E} = -(kp/y^3){\bf i}$, where $p$ is
  the magnitude of the dipole moment.

  \begin{solution}
    The force on the point only exists in the $x$ direction by
    symmetry. But the $x$ components are in direction of $-{\bf i}$.

    So,
    $$F_{net} = -\frac{2kq\ dq}{y^2 + a^2} \cdot \cos \theta {\bf i}
    = \frac{-2kq\  dq}{y^2+a^2}\frac{a}{\sqrt{y^2 + a^2}} {\bf i}$$

    Therefore
    $${\bf E} = \frac{ {\bf F} }{ dq } = - \frac{kp}{ (y^2+a^2)^{3/2}
    } {\bf i}$$

    If $y >> a$ then $(y^2 + a^2)^{3/2} \approx y^3$.
    So if $y >> a$, then ${\bf E} \approx - \frac{kp}{y^3} {\bf i}$.
  \end{solution}

  \section{Continuous Charge Distributions}

  % Example on page 637 of 19-3
  \question[8] Use Gauss' law to compute the electric field near an
  infinite line charge.

  \begin{solution}
    Consider a cyclinder of length $L$ and radius $r$ with uniform
    charge density $\lambda$,
    $$\phi_{net} = \int E_n dA = \frac{1}{\epsilon_0} Q_{inside}$$
    And,
    $$\int E_n dA = E_r \int dA = \frac{\lambda L}{\epsilon_0}$$
    The area of the cyclindrical surface is $2\pi r L$,
    $$E_r 2\pi r L = \frac{\lambda L}{\epsilon_0}$$
    $$E_r = \frac{1}{2\pi \epsilon_0} \frac{\lambda}{r} = 2k \frac{\lambda}{r}$$
  \end{solution}

  \section{Electric Potential}

  % 20.13, 25, 40*, 43, 61ab

  % 20.25
  \question[3] Find the maximum net charge that can be placed on a
  spherical conductor of radius 16 cm before dielectric breakdown of
  the air occurs.

  \begin{solution}
    $$\frac{\sigma}{\epsilon_0} = 3 MN/C$$
    $$\sigma = 2.66 \times 10^{-5}$$
    $$Q = 4\pi r^2 \sigma = 8.55 \mu C$$
  \end{solution}

  % 20.21
  \question[2] If $V(x) = 2000 + 3000x$ where $V$ is in volts and $x$
  is in meters, what is $E_x$?

  \begin{solution}
    $$-3000 V/m$$
  \end{solution}

  % 20.13
  \question[2] Point charges $q_1$, $q_2$, and $q_3$ are at the
  corners of an equilateral triangle of side 2.5 m. Find the
  electrostatic potential energy of this charge distribution if $(q_1
  = q_2 = q_3 = 4.2 \mu C)$. 

  \begin{solution}
    $$U = 0.19 J$$
  \end{solution}

  \section{Currents and Capacitance}

  % 21: 10, 34, 44
  % 22: 26, 27, 29
  
  % 21.10
  \question[3] A $3-\mu F$ capacitor is charged to 100 V. How much
  energy did it take to bring this capacitor to this charge level?

  \begin{solution}
    $$E = \frac{1}{2}C V^2 = 0.015 J$$
  \end{solution}

  \question (\totalpoints\ points) A parallel-plate capacitor of plate
  area $A$ and separation $x$ is given a charge $Q$ and is then
  removed from the charging source. 
  \begin{parts}
    \part[2] Find the stored electrostatic energy as a function of
    $x$.
    \begin{solution}
      $$C = \epsilon_0 A/x$$
      $$U = \frac{1}{2} \frac{x}{\epsilon_0 A} Q^2$$
    \end{solution}
    \part[2] Find the increase in energy $dU$ due to an increase in
    plate separation $dx$ from $dU = (dU/dx) dx$. 
    \begin{solution}
      $$\frac{dU}{dx} = \frac{Q^2}{2\epsilon_0 A}$$
      $$dU = \frac{Q^2}{2\epsilon_0 A} dx$$
    \end{solution}
    \part[2] If $F$ is the force exerted by one plate on the other,
    the work needed to move one plate a distance $dx$ is $F dx =
    dU$. Show that $F = Q^2/2\epsilon_0 A$.
    \begin{solution}
      $$F = \frac{dU}{dx} = \frac{Q^2}{2\epsilon_0 A}$$
    \end{solution}
    \part[3] Show that the force in part (c) equals $\frac{1}{2} EQ$
    where $Q$ is the charge on one plate and $E$ is the electric field
    between the plates. Discuss the reason for the factor of
    $\frac{1}{2}$ in this.
    \begin{solution}
      $$E = \frac{\sigma}{\epsilon_0} = \frac{Q}{A\epsilon_0}$$
      So $\frac{1}{2} EQ^2 = \frac{Q^2}{2A\epsilon_0} = F$. 
      
      Why? Force on a plate $= \int_0^L E \sigma dx =
      \frac{\sigma^2}{2\epsilon_0} A = \frac{Q^2}{2A\epsilon_0}$. The
      $E$ in this integral is due to the other plate. 

      The net field in the plate is zero, the charges don't act on
      themsleves, so only the electric field of the other plate needs
      to be considered.
    \end{solution}
  \end{parts}

  \section{DC Circuits}

  % 22: 56, 58, 
  % 23: 13, 14, 24, 41
  
  % 22.27
  \question (\totalpoints\ points) A battery with 12-V emf has a
  terminal voltage of 11.4V when it delivers a current of 20 A to the
  starter of a car. 
  \begin{parts}
    \part[1] How much power is delivered by the emf?
    \begin{solution}
      $$P = VI = 240 W$$
    \end{solution}
    \part[1] How much power is delivered to the starter?
    \begin{solution}
      228 W
    \end{solution}
    \part[1] By how much does the chemical energy of the battery
    decrease when it delivers a current of 20 A to the starter for 3
    min?
    \begin{solution}
      $$240 W \cdot 180 s = 43.2 kJ$$
    \end{solution}
    \part[1] How much eat is developed in the battery when it delivers
    a current of 20 A for 3 min?
    \begin{solution}
      $$(240 - 228) \cdot 180 s = 2.17kJ$$
    \end{solution}
  \end{parts}

  % 23. 24
  \question (\totalpoints\ points) A sick car battery with an emf of
  11.4 V and an internal resitance of 0.01 $\Omega$ is connected to a
  load of 2.0 $\Omega$. To help the ailing battery a second battery
  with an emf of 12.6 V and an internal resistance of $0.01 \Omega$ is
  connected by jumper cables to the terminals of the first battery.
  
  \begin{parts}
    \part[2] Draw a diagram of this circuit.
    \begin{solution}
      
    \end{solution}

    \part[3] Find the current in each part of the circuit.
    \begin{solution}
      
    \end{solution}

  \end{parts}


  % 23.41
  \question (\totalpoints\ points) Consider the circuit:
  
  \includegraphics[scale=0.75]{circuit23-41.pdf}

  From your knowledge of how capacitors behave find 
  \begin{parts}
    \part[2] The initial current through the battery just after the
    switch is closed.
    \begin{solution}
      $$120V - 1.2 M\Omega  I = 0$$
      $$I = 0.10 mA$$
    \end{solution}
    \part[2] The steady-state current through the batter when the
    switch has been closed for a long time.
    \begin{solution}
      $$120V - 600K\Omega I - 1.2 M\Omega I = 0$$
      $$I = 67 \mu A$$
    \end{solution}
    \part[2] The maximum voltage across the capacitor.
    \begin{solution}
      $$67 \mu A 600 k \Omega = 40.2 V$$
    \end{solution}
  \end{parts}

  \section{The Magnetic Field}
  % 24: 4, 15, 31, 42, 56

  % 24.31
  \question[2] Consider a wire segment in the $xy$ plane that consists
  of a 3 cm segment parallel to the $x$ axis followed by a 4 cm
  segment parallel to the $y$ axis. Call the end point of the first
  segment $a$ and the endpoint of the second $b$. There is a magnetic
  field ${\bf B} = 1.2 T {\bf k}$. Find the total force on the wire.

  \begin{solution}
    $${bf B}_{at\ wire} = 1.8A (3\ cm\ {\bf i} + 4\ cm\ {\bf j})
    \times 1.2\ {\bf k} T$$
    $$ = (-0.0864\ {\bf i} - 0.0648\ {\bf j})N$$
  \end{solution}

  % 24.42 
  \question (\totalpoints\ points) Beryllium has a density of 1.83
  g/cm$^3$ and a molar mass of 9.01 g/mol. A slab of beryllium of
  thickness 1.4 mm and width 1.2 cm carries a current of 3.75 A in a
  region in which there is a magnetic field of magnitude 1.88 T
  perpendicular to the slab. The Hall voltage is measured to be $0.130
  \mu V$.

  \begin{parts}
    \part[2] Calculate the number density of the charge carriers
    \begin{solution}
      $$n = \frac{IB}{e+V_H} = 242 \times 10^{27} l/m^3$$
    \end{solution}
    \part[2] Calculate the number density of atoms in beryllium
    \begin{solution}
      $$n = \frac{N_a \rho}{\mu} = 122 \times 10^{21} \frac{atoms}{cm^3}$$
    \end{solution}
    \part[2] How many free electrons are there per atom of beryllium?
    \begin{solution}
      2 free electrons.
    \end{solution}
  \end{parts}


  \section{Sources of Magnetic Field}
  % 25: 4, 15, 16, 31, 34, 56

  % 25.15
  \question[2] Consider two long straight wires in the $xy$ plane and
  parallel to the $x$ axis. One wire is at $y = -6 cm$ and the other
  is at $y = +6 cm$. The current in each wire is 20 A. If the currents
  are in the negative $x$ direction, find ${\bf B}$ at $y = -3 cm$.

  \begin{solution}
    At $y = -3 cm$, ${\bf B} = -0.89 \mu T {\bf k}$ using the
    Biot-Savart Law.
  \end{solution}

  \question[4] Consider a current 8 A into the paper, and another 8 A
  out of the paper. Consider three paths: $C_1$ around just the one
  into the paper, $C_2$ around the current out and $C_3$ around
  both. Find $\int {\bf B} \cdot d{\bf l}$ for each path
  indicated. Which path, if any, can be used to find {\bf B} at some
  point due to these currents?

  \begin{solution}
    Using Ampere's Law
    $$C_1: 8 \mu_0$$
    $$C_2: -8 \mu_0$$
    $$C_3: 0$$
    None as ${\bf B}$ is not uniform, so there is no symmetry.
  \end{solution}

  \section{Induction}
  % 26: 3*, 14*, 18, 54, 66*, 67*, 79*

  % 26.66
  \question[2] Consider a conducting rod of length
  $l$ moving through a magnetic field ${\bf B}$ perpendicular to it's
  length while sliding on two conducting parallel rails perpendicular
  to the rod. Suppose the end that the rod is moving away from is
  connected by a resistor $R$. Let the mass of the rod be $m$.What is
  the magnitude of the magnetic force acting on the rod in terms of
  the magnetic field $B$, the length $l$, the velocity $v$ and the
  resistance $R$?
  \begin{solution}
    Recall $\mathcal{E} = Blv$.
    $$F = IBl = \frac{\mathcal{E}}{R} Bl = \frac{Blv}{R}Bl = \frac{B^2l^2v}{R}$$
  \end{solution}
  
  % 26.14
  \question[2] A circular coil of 300 turns and radius 5.0 cm is
  connected to a current integrator. The total resistance of the
  circuit is $20 \Omega$. The plane of the coil is originally aligned
  perpendicular to the earth's magnetic field at some point. When the
  coil is rotated through $90^\circ$ the charge that passes through
  the current integrator is measured to be $9.4\ \mu c$. Calculate the
  magnitude of the earth's magnetic field at that point.

  \begin{solution}
    $$\Delta \Phi = NBA$$
    $$I = \frac{1}{R} \frac{d\Phi}{dt}$$
    $$Q = \frac{1}{R} \int d\Phi = \frac{\Delta \Phi}{R}$$
    $$\Delta \Phi = 0.188 mWb$$
    $$B = 0.080 mT$$
  \end{solution}
  
  \question[2] Two parallel loops have their planes parallel. As
  viewed from loop A to loop B there is a counterclockwise current in
  loop A. Give the direction of the current in loop B and state
  whether the loops attract or repel each other if the current in loop
  A is (a) increasing and (b) decreasing.

  \begin{solution}
    Use Lenz's law
  \end{solution}
  

  \section{AC Circuits}
  % 28: 8, 11, 29, 75

  % 28.29
  \question (\totalpoints\ points) A transformer has 400 turns in the
  primary and 8 turns in the secondary. 
  \begin{parts}
    \part[1] Is this a step-up or step-down transformer?
    \part[1] If the primrary is connected across 120 V rms what is the
    open-circuit voltage across the secondary?
    \part[2] If the primrary current is 0.1 A what is the secondary
    current, assuming negligible magnetization current and no power loss?
  \end{parts}

  \section{Maxwell's Equations}
  % 29: 1, 7, 16, 40, 42, 45

  \section{Relativity}
  % 34: 3, 14, 15, 18, 51, 75

  \section{Light}
  % 30: 8, 9, 14, 50, 53, 54

  \section{Interference and Diffraction}
  % 33: 8, 10, 27, 54, 55, 57


  \section{Optics}
  % 32:
  % 33:

\end{questions}

\end{document}
