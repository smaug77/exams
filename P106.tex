% !TEX TS-program = pdflatex
% !TEX encoding = UTF-8 Unicode

\documentclass[12pt,addpoints]{exam} % use larger type; default would be 10pt
\usepackage[utf8]{inputenc} % set input encoding (not needed with XeLaTeX)
\usepackage{listings}
\usepackage{amssymb} % for \nmid


%%% Examples of Article customizations
% These packages are optional, depending whether you want the features they provide.
% See the LaTeX Companion or other references for full information.

%%% PAGE DIMENSIONS
\usepackage{geometry} % to change the page dimensions
\geometry{a4paper} % or letterpaper (US) or a5paper or....
% \geometry{margins=2in} % for example, change the margins to 2 inches all round
% \geometry{landscape} % set up the page for landscape
%   read geometry.pdf for detailed page layout information


% \usepackage[parfill]{parskip} % Activate to begin paragraphs with an empty line rather than an indent

%%% PACKAGES
\usepackage{booktabs} % for much better looking tables
\usepackage{array} % for better arrays (eg matrices) in maths
\usepackage{paralist} % very flexible & customisable lists (eg. enumerate/itemize, etc.)
\usepackage{verbatim} % adds environment for commenting out blocks of text & for better verbatim
\usepackage{subfig} % make it possible to include more than one captioned figure/table in a single float
\usepackage{amsthm} % for proof environment
% These packages are all incorporated in the memoir class to one degree or another...
\usepackage{amsfonts} % allows use of mathbb 
\usepackage{amsmath}
\usepackage{multicol}
\usepackage{graphicx} % support the \includegraphics command and options


%%% HEADERS & FOOTERS
%\usepackage{fancyhdr} % This should be set AFTER setting up the page geometry
%\pagestyle{fancy} % options: empty , plain , fancy
%\renewcommand{\headrulewidth}{0pt} % customise the layout...
%\lhead{}\chead{}\rhead{}
%\lfoot{}\cfoot{\thepage}\rfoot{}

%%% SECTION TITLE APPEARANCE
\usepackage{sectsty}
\allsectionsfont{\sffamily\mdseries\upshape} % (See the fntguide.pdf for font help)
% (This matches ConTeXt defaults)

%%% ToC (table of contents) APPEARANCE
\usepackage[nottoc,notlof,notlot]{tocbibind} % Put the bibliography in the ToC
\usepackage[titles,subfigure]{tocloft} % Alter the style of the Table of Contents
\renewcommand{\cftsecfont}{\rmfamily\mdseries\upshape}
\renewcommand{\cftsecpagefont}{\rmfamily\mdseries\upshape} % No bold!
\newcommand{\Dgrad}[1]{\textsf{grad}\,#1\;}
\newcommand{\del}[1] {\mathrm{d}#1\ }
\newcommand{\Q}{\mathbb{Q}}
\newcommand{\F}{\mathbb{F}}
\newcommand{\C}{\mathbb{C}}
\newcommand{\Z}{\mathbb{Z}}
\newcommand{\R}{\mathbb{R}}
\newcommand{\N}{\mathbb{N}}
\newcommand{\PP}{\mathbb{P}}
\newcommand{\arcsec}{\mbox{arcsec}}
\newcommand{\arccsc}{\mbox{arccsc}}
\newcommand{\arccot}{\mbox{arccot}}
\newcommand{\csch}{\mbox{csch}}
\newcommand{\sech}{\mbox{sech}}

%%% NEW ENVIRONMENTS
\newtheorem{theorem}{Theorem}
\newtheorem{definition}[theorem]{Definition}
\newtheorem{example}[theorem]{Example}
\newtheorem{proposition}[theorem]{Proposition}
\newtheorem{corollary}[theorem]{Corollary}
\newtheorem{lemma}[theorem]{Lemma}

%%% END Article customizations

%%% The "real" document content comes below...


\usepackage{color}
\lstset{numbers=left}
\begin{document}

\printanswers

\title{Physics 106}
\maketitle



\begin{center}
    This exam has \numquestions\ questions for a total of \numpoints\
    points. You have {\bf 2 Hours} to complete this exam.
\end{center}

% 14 weeks:
% 1: 18 (8.5 points)
% 2: 19 (17)
% 3: 20 (25.5)
% 4: 21-22 (34)
% 5: 22-23 (42.5)
% 6: 24 (53)
% 7: 25 (61.5)
% 8: 26 (70)
% 9: 28 (78.5)
% 10: 29 (87)
% 11: 34 (95.5)
% 12: 30 (104)
% 13: 33 (112.5)
% 14: 31-32 (121)
% 8.5 points per week.
\begin{questions}

  \section{Discrete Charge Distributions}
  
  % 18.39
  \question[8] An electric dipole consists of a positive charge $q$ on
  the $x$ axis at $x = a$ and a negative charge $-q$ on the $x$ axis
  at $x = -a$. Find the magnitude and direction of the electric field
  at a point $y$ on the $y$ axis, and show that for $y >> a$, the
  field is approximately ${\bf E} = -(kp/y^3){\bf i}$, where $p$ is
  the magnitude of the dipole moment.

  \begin{solution}
    The force on the point only exists in the $x$ direction by
    symmetry. But the $x$ components are in direction of $-{\bf i}$.

    So,
    $$F_{net} = -\frac{2kq\ dq}{y^2 + a^2} \cdot \cos \theta {\bf i}
    = \frac{-2kq\  dq}{y^2+a^2}\frac{a}{\sqrt{y^2 + a^2}} {\bf i}$$

    Therefore
    $${\bf E} = \frac{ {\bf F} }{ dq } = - \frac{kp}{ (y^2+a^2)^{3/2}
    } {\bf i}$$

    If $y >> a$ then $(y^2 + a^2)^{3/2} \approx y^3$.
    So if $y >> a$, then ${\bf E} \approx - \frac{kp}{y^3} {\bf i}$.
  \end{solution}

  \section{Continuous Charge Distributions}

  % Example on page 637 of 19-3
  \question[8] Use Gauss' law to compute the electric field near an
  infinite line charge.

  \begin{solution}
    Consider a cyclinder of length $L$ and radius $r$ with uniform
    charge density $\lambda$,
    $$\phi_{net} = \int E_n dA = \frac{1}{\epsilon_0} Q_{inside}$$
    And,
    $$\int E_n dA = E_r \int dA = \frac{\lambda L}{\epsilon_0}$$
    The area of the cyclindrical surface is $2\pi r L$,
    $$E_r 2\pi r L = \frac{\lambda L}{\epsilon_0}$$
    $$E_r = \frac{1}{2\pi \epsilon_0} \frac{\lambda}{r} = 2k \frac{\lambda}{r}$$
  \end{solution}

  \section{Electric Potential}

  % 20.13, 25, 40*, 43, 61ab

  % 20.25
  \question[3] Find the maximum net charge that can be placed on a
  spherical conductor of radius 16 cm before dielectric breakdown of
  the air occurs.

  \begin{solution}
    $$\frac{\sigma}{\epsilon_0} = 3 MN/C$$
    $$\sigma = 2.66 \times 10^{-5}$$
    $$Q = 4\pi r^2 \sigma = 8.55 \mu C$$
  \end{solution}

  % 20.21
  \question[2] If $V(x) = 2000 + 3000x$ where $V$ is in volts and $x$
  is in meters, what is $E_x$?

  \begin{solution}
    $$-3000 V/m$$
  \end{solution}

  % 20.13
  \question[2] Point charges $q_1$, $q_2$, and $q_3$ are at the
  corners of an equilateral triangle of side 2.5 m. Find the
  electrostatic potential energy of this charge distribution if $(q_1
  = q_2 = q_3 = 4.2 \mu C)$. 

  \begin{solution}
    $$U = 0.19 J$$
  \end{solution}

  \section{Currents and Capacitance}

  % 21: 10, 34, 44
  % 22: 26, 27, 29, 56, 58
  
  % 21.10
  \question[3] A $3-\mu F$ capacitor is charged to 100 V. How much
  energy did it take to bring this capacitor to this charge level?

  \begin{solution}
    $$E = \frac{1}{2}C V^2 = 0.015 J$$
  \end{solution}

  
    

\end{questions}

\end{document}
